\section{Program Overview}

\subsection{General Description}

The goal of the project was to design, implement and operate a MD computer program. To fulfill these the following steps have been done:
\begin{itemize}
\item MD code design 
\item Correct selection and writing of code subroutines
\item MD program assembly
\item Compilation, debugging and testing
\item Operation of the program
\item Analysis of the results obtained
\item Assertion of the quality of the results
\end{itemize}

The program coding language is C++ and has been compiled with Visual Studio 2010 \textregistered. For the visualization of the different results, Matlab R2010b \textregistered was used to plot the properties and to display the atom movements.

In the program, the Velocity Verlet algorithm is used for integration and the Lennard-Jones potential is used for the force calculations.

The following properties are calculated by the program:

\begin{itemize}
\item Kinetic Energy
\item Potential Energy
\item Total Energy
\item Cohesive Energy
\item Mean Square Displacement (MSD)
\item Internal Pressure 
\item Temperature 
\item Debye Temperature 
\item Specific Heat 
\item Diffusion Coefficient
\end{itemize}

The runtime as well as the average time per time step, as well as the average of the properties specified previously, are displayed at the end of the simulation.

\subsection{Program Capabilities}

The program capabilities are presented in the following sections. How to use the GUI is presented in detail in the Manual section of this document.

\subsubsection{Material}

The program can simulate 38 different materials that can be chosen in the tab ''Material'' in the GUI. The materials that have been used for this report are:

\begin{itemize}
\item Argon (Ar) 
\item Silver (Ag)
\end{itemize}

\subsubsection{Structure}

In the ''Structure'' tab, the following properties, aside from the dimensions of the system, may be freely changed by the user:

\begin{itemize}
\item Lattice constant (in \AA) 
\item Cutoff distance (in \AA)
\item Sigma (in \AA) 
\item Epsilon (in eV)
\end{itemize}

The available crystal structures are:
\begin{itemize}
\item Face-centered cubic (FCC) 
\item Body-centered cubic (BCC)
\item Diamond (DIA)
\end{itemize}

\subsubsection{Calculations}

In the ''Calculations'' tab the following simulation specific values may be changed:
\begin{itemize}
\item Run for \# timesteps 
\item Timestep size (in fs)
\item Start calculations after \# timesteps
\item Store data every \# timesteps
\end{itemize}

By clicking the "Run from previous simulation" button, a .txt-file from a previous simulation can be chosen from which to run the new simulation.

By ticking the "Include visualisation", a file will be written containing the atoms position at different timesteps which will be used in Matlab to visualize the atom displacement.

Ticking "Thermostat" includes an Andersen thermostat for which the collision rate may be specified by the user.

The Temperature in K and whether or not to use periodic boundary conditions for each direction can also be specified.

