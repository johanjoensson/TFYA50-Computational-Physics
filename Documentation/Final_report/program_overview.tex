\section{Program overview}

\subsection{General Description}

The project goal is to design, implement and operate a Molecular Dynamics (MD) program. Thus the different step had been done for he completion of the project:

\begin{itemize}
\item MD coding design flow ; 
\item correct selection and writing of code subroutines ;
\item MD program assembling ;
\item compiling, debugging and testing ;
\item operating the program ;
\item analyze results obtained with the program ;
\item assert the quality of the results.
\end{itemize}

The program coding language is C++ and has been compiled with Visual Studio 2010 \textregistered. For the visualization of the different results, Matlab R2010b \textregistered was used to draw the curve and display the atoms movements.

Our program uses the Verlet list, the Velocity Verlet algorithm and the Lennard Jones potential for the calculation.

The simulation gives the following results at the end of each simulation:

\begin{itemize}
\item Kinetic Energy ;
\item Potential Energy ;
\item Total Energy ;
\item Cohesive Energy ;
\item Mean Square Displacement (MSD) ;
\item Internal Pressure ; 
\item Temperature ; 
\item Debye Temperature ; 
\item Specific Heat ; 
\item Diffusion Coefficient.
\end{itemize}

And the time duration of the simulation as well as the average time per time step are displayed at the end of the simulation.

\subsection{Program Capabilities}

The program displays its capabilities in 4 different tab presented below.

\subsubsection{Material}

The program can simulate 38 different materials that can be chosen in the tab Material and the main materials that have been used for simulation are:

\begin{itemize}
\item Argon (Ar) ; 
\item Silver (Ag).
\end{itemize}

As a reminder, the crystal structure of each material are notified.

\subsubsection{Structure}

Once the materials has been chosen, the following parameters are displayed and can be changed in the tab Structure:

\begin{itemize}
\item Lattice constant (in \AA) ; 
\item Cutoff distance (in \AA) ;
\item Sigma (in \AA) ; 
\item Epsilon (in eV).
\end{itemize}

Then it is possible to choose the number of unit cells we want to use for the simulation in every direction, thus simulation of bulk and surface is possible.

The crystalline structure can also be chosen among the 3 coded ones:

\begin{itemize}
\item Face-centered cubic (FCC) ; 
\item Body-centered cubic (BCC);
\item Diamond (DIA).
\end{itemize}

Since Argon and Silver both have a FCC structures, it is the one that we will be used for the simulation.

\subsubsection{Calculations}

This tab allows us to control the length of the simulation and the time of data storing using:

\begin{itemize}
\item Run for \# timesteps ; 
\item Timestep size (in fs) ;
\item Start calculations after \# timesteps ; 
\item Store data every \# timesteps.
\end{itemize}

By clicking on the "Run from previous simulation" button, we can select a .txt file from which the simulation will run from by using the different material properties and atom's positions and velocities written in it.

By ticking the "Include visualisation", a file will be written containing the atoms position at different timesteps which will be used in Matlab to visualize the atom displacement.

Ticking "Thermostat" includes an Andersen thermostat for which we can choose the collision rate.

The Temperature (in K) can also be defined and the periodic boundary conditions in every direction.

\subsubsection{Results}

In order to start the simulation, we have to clicl on the button 'Start Simulation". The window on the right display a summary of the simulation choice at the beginning of the simulation.
