\section{Training plan}

\subsection{Group training}
In order for all the members of the group to be able to contribute to the
project it might be necessary to have some training in the different tools
used by the group. Examples of this includes the C++ programming language, the
version control method chosen by the group (git, SVN etc.) and the graphics
library OpenGL used for the non compulsory visualisation of the results of the
simulation. The group will also receive training in MD-simualtions, the group
will be given lectures and do some laborations on MD-simulation during the
project.

In the course, TFYA50, that the project is a part of there will also be
lectures in entrepeneurship, some case studies and an exam. All given by
Magnus Klofsten from IEI.

The MD-laborations are mandatory for all group members and are given to
prepare the group for the project.

\subsection{Client training}
All information necessary for the client to use the finished product will be
given in the technical documentation written during the project. No other
training for the client is necessary.
